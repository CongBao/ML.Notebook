%!TEX root = ../notebook.tex
% Chapter 6

\chapter{Linear Regression}
\label{chapter6}

\section{Non-Probability Model}
\label{section6.1}

We want to fit a linear line on the data so we introduce the linear relation for one datapoint $\vec{x}_i$ and the corresponding result $y_i$:
\begin{align*}
	y_i=\vec{\ph}\T\vec{x}_i+\ep_i
\end{align*}
where $\vec{x}_i$ is a data point with $1$ as the first item and $\vec{\ph}$ is the parameter vector with the first item as a bias term. $\ep_i$ is a noise. For a whole dataset:
\begin{align*}
	\vec{y}=\mat{X}\T\vec{\ph}+\vec{\ep}
\end{align*}
where $\mat{X}\in\bR^{D\times N}$. We can apply a squared error loss on the noise term:
\begin{align*}
	\cL(\vec{\ph})=\sum_{i=1}^N\ep_i^2=\sum_{i=1}^N\left(y_i-\vec{\ph}\T\vec{x}_i\right)^2
\end{align*}
Taking $\grad\cL(\vec{\ph})=0$ and we have the solution:
\begin{align*}
	\vec{\ph}^*=\left(\mat{X}\mat{X}\T\right)\inv\mat{X}\vec{y}
\end{align*}



\section{Probability Model}
\label{section6.2}

We choose normal distribution over world $w_i$ with mean as a linear function of data $\vec{x}_i$ and constant variance:
\begin{align*}
	\P{w_i|\vec{x}_i,\vec{\th}}=\Norm_{w_i}[\vec{\ph}\T\vec{x}_i,\si^2]
\end{align*}
where $\vec{x}_i$ is a data point with $1$ as the first item and $\vec{\ph}$ is the parameter vector with the first item as a bias term. For a whole dataset, the probability is the product of these individual distributions:
\begin{align*}
	\P{\vec{w}|\mat{X},\vec{\th}}=\Norm_{\vec{w}}[\mat{X}\T\vec{\ph},\si^2\mat{I}]
\end{align*}
where $\mat{X}\in\bR^{D\times N}$, $\vec{\ph}\in\bR^{D\times1}$. Hence the distribution is a multivariate normal distribution with mean a vector in $\bR^N$ and covariance a diagonal matrix in $\bR^{N\times N}$ with $\si^2$ on its diagonal. Learning with maximum likelihood: $\hat{\vec{\th}}=\argmax_{\vec{\th}}\P{\vec{w}|\mat{X},\vec{\th}}=\argmax_{\vec{\th}}\log\P{\vec{w}|\mat{X},\vec{\th}}$, taking derivative and set result to zero and re-arrange:
\begin{align*}
	\hat{\vec{\ph}}&=\left(\mat{X}\mat{X}\T\right)\inv\mat{X}\vec{w} \\
	\hat{\si}^2&=\frac{\left(\vec{w}-\mat{X}\T\vec{\ph}\right)\T\left(\vec{w}-\mat{X}\T\vec{\ph}\right)}{\mat{I}}
\end{align*}



\section{Bayesian Regression}
\label{section6.3}

We already got the likelihood term $\P{\vec{w}|\mat{X},\vec{\th}}=\Norm_{\vec{w}}[\mat{X}\T\vec{\ph},\si^2\mat{I}]$. Now we give it a prior distribution $\P{\vec{\ph}}=\Norm_{\vec{\ph}}[\vec{0},\si_p^2\mat{I}]$ on the mean $\vec{\ph}$. We can use Bayes' rule to calculate the posterior distribution of $\vec{\ph}$:
\begin{align*}
	\P{\vec{\ph}|\mat{X},\vec{w}}&=\frac{\P{\vec{w}|\mat{X},\vec{\ph}}\P{\vec{\ph}|\mat{X}}}{\P{\vec{w}|\mat{X}}} \\
	&=\Norm_{\vec{\ph}}\left[\frac{1}{\si^2}\mat{A}\inv\mat{X}\vec{w},\mat{A}\inv\right]
\end{align*}
where
\begin{align*}
	\mat{A}&=\frac{1}{\si^2}\mat{X}\mat{X}\T+\frac{1}{\si_p^2}\mat{I} \\
	\mat{A}\inv&=\si_p^2\mat{I}_D-\si_p^2\mat{X}\left(\mat{X}\T\mat{X}+\frac{\si^2}{\si_p^2}\mat{I}_N\right)\inv\mat{X}\T
\end{align*}
With the posterior, we can do Bayesian inference that
\begin{align*}
	\P{w^*|\vec{x}^*,\mat{X},\vec{w}}&=\int\P{w^*|\vec{x}^*,\vec{\ph}}\P{\vec{\ph|\mat{X},\vec{w}}}d\vec{\ph} \\
	&=\int\Norm_{w^*}[\vec{\ph}\T\vec{x}^*,\si^2]\Norm_{\vec{\ph}}\left[\frac{1}{\si^2}\mat{A}\inv\mat{X}\vec{w},\mat{A}\inv\right]d\vec{\ph} \\
	&=\Norm_{w^*}\left[\frac{1}{\si^2}\vec{x}^*\T\mat{A}\inv\mat{X}\vec{w},\vec{x}^*\T\mat{A}\inv\vec{x}^*+\si^2\right]
\end{align*}
For the variance $\si^2$, we optimize the marginal likelihood that
\begin{align*}
	\P{\vec{w}|\mat{X},\si^2}&=\int\P{\vec{w}|\mat{X},\vec{\ph},\si^2}\P{\vec{\ph}}d\vec{\ph} \\
	&=\int\Norm_{\vec{w}}[\mat{X}\T\vec{\ph},\si^2\mat{I}]\Norm_{\vec{\ph}}[\vec{0},\si_p^2\mat{I}]d\vec{\ph} \\
	&=\Norm_{\vec{w}}[\vec{0},\si_p^2\mat{X}\T\mat{X}+\si^2\mat{I}]
\end{align*}

\section{Non-linear Regression}

\paragraph{Basic Idea}

One can make a non-linear function from a linear weighted sum of non-linear basis functions. We have known the linear regression:
\begin{align*}
	\P{w_i|\vec{x}_i,\vec{\th}}=\Norm_{w_i}[\vec{\ph}\T\vec{x}_i,\si^2]
\end{align*}
Then the non-linear regression is:
\begin{align*}
	\P{w_i|\vec{x}_i,\vec{\th}}=\Norm_{w_i}[\vec{\ph}\T\vec{z}_i,\si^2]
\end{align*}
where $\vec{z}_i=f(\vec{x}_i)$ with $f$ as some non-linear basis functions.

\paragraph{Polynomial Regression}

The non-linear basis function $f(\bullet)$ can be a polynomial function, for example, $\vec{z}_i=f(\vec{x}_i)=[1\ x_i\ x_i^2\ x_i^3]\T$. In this case, we can obtain the estimated parameter similar to linear regression but with $\mat{Z}=f(\mat{X})$ instead of $\mat{X}$:
\begin{align*}
	\hat{\vec{\ph}}&=\left(\mat{Z}\mat{Z}\T\right)\inv\mat{Z}\vec{w} \\
	\hat{\si}^2&=\frac{\left(\vec{w}-\mat{Z}\T\vec{\ph}\right)\T\left(\vec{w}-\mat{Z}\T\vec{\ph}\right)}{\mat{I}}
\end{align*}

\paragraph{Radial Basis Functions (RBF)} We can apply a Gaussian radial basis function on the data that $\vec{z}=\e{-\pfrac{(\vec{x}-\vec{\al})^2}{\la}}$.

\paragraph{Arc Tan Functions} We can apply an arc tan function on the data that $\vec{z}=\arctan(\la\vec{x}-\vec{\al})$.

\section{Kernel Trick \& Gaussian Processes}

With any linear or non-linear basis function and the Bayesian inference, we can get the predictive distribution that
\begin{align*}
	\P{w^*|\vec{z}^*,\mat{Z},\vec{w}}&=\int\P{w^*|\vec{z}^*,\vec{\ph}}\P{\vec{\ph|\mat{Z},\vec{w}}}d\vec{\ph} \\
	&=\Norm_{w^*}\left[\frac{1}{\si^2}\vec{z}^*\T\mat{A}\inv\mat{Z}\vec{w},\vec{z}^*\T\mat{A}\inv\vec{z}^*+\si^2\right] \\
	&=\Norm_{w^*}\Bigg[\frac{\si_p^2}{\si^2}\vec{z}^*\T\mat{Z}\vec{w}-\frac{\si_p^2}{\si^2}\vec{z}^*\T\mat{Z}\left(\mat{Z}\TT\mat{Z}+\frac{\si^2}{\si_p^2}\mat{I}\right)\inv\mat{Z}\TT\mat{Z}\vec{w},\\
	&\qquad\qquad\qquad\si_p^2\vec{z}^*\T\vec{z}^*-\si_p^2\vec{z}^*\T\mat{Z}\left(\mat{Z}\TT\mat{Z}+\frac{\si^2}{\si_p^2}\mat{I}\right)\inv\mat{Z}\TT\vec{z}^*+\si^2\Bigg]
\end{align*}
Notice that the final equation does not need the data itself, but just dot products between data items of the form $\vec{z}_i\TT\vec{z}_j$. So, we take data $\vec{x}_i$ and $\vec{x}_j$ pass through non-linear function to create $\vec{z}_i$ and $\vec{z}_j$ and then take dot products of different $\vec{z}_i\TT\vec{z}_j$.

The kernel trick is that, we can define a kernel function that does all of this together. Instead of compute $\vec{z}$ explicitly, which can be very high or infinite dimension, we takes data $\vec{x}_i$ and $\vec{x}_j$ and returns a value for dot product $\vec{z}_i\TT\vec{z}_j$. Then, the predictive distribution above can be written as
\begin{align*}
	\P{w^*|\vec{z}^*,\mat{Z},\vec{w}}&=\int\P{w^*|\vec{z}^*,\vec{\ph}}\P{\vec{\ph|\mat{Z},\vec{w}}}d\vec{\ph} \\
	&=\Norm_{w^*}\Bigg[\frac{\si_p^2}{\si^2}\mat{K}[\vec{x}^*,\mat{X}]\vec{w}-\frac{\si_p^2}{\si^2}\mat{K}[\vec{x}^*,\mat{X}]\left(\mat{K}[\mat{X},\mat{X}]+\frac{\si^2}{\si_p^2}\mat{I}\right)\inv\mat{K}[\mat{X},\mat{X}]\vec{w},\\
	&\qquad\qquad\qquad\si_p^2\mat{K}[\vec{x}^*,\vec{x}^*]-\si_p^2\mat{K}[\vec{x}^*,\mat{X}]\left(\mat{K}[\mat{X},\mat{X}]+\frac{\si^2}{\si_p^2}\mat{I}\right)\inv\mat{K}[\mat{X},\vec{x}^*]+\si^2\Bigg]
\end{align*}
where the notation $\mat{K}[\mat{X},\mat{X}]$ represents a matrix of dot products where element $(i,j)$ is given by $k[\vec{x}_i,\vec{x}_j]$. Similarly, for the variance $\si^2$, we optimize the marginal likelihood that
\begin{align*}
\P{\vec{w}|\mat{Z},\si^2}&=\int\P{\vec{w}|\mat{Z},\vec{\ph},\si^2}\P{\vec{\ph}}d\vec{\ph} \\
&=\int\Norm_{\vec{w}}[\mat{Z}\T\vec{\ph},\si^2\mat{I}]\Norm_{\vec{\ph}}[\vec{0},\si_p^2\mat{I}]d\vec{\ph} \\
&=\Norm_{\vec{w}}[\vec{0},\si_p^2\mat{Z}\T\mat{Z}+\si^2\mat{I}] \\
&=\Norm_{\vec{w}}[\vec{0},\si_p^2\mat{K}[\mat{X},\mat{X}]+\si^2\mat{I}]
\end{align*}

Some examples of kernel functions:
\begin{enumerate}
	\item linear: $k[\vec{x}_i,\vec{x}_j]=\vec{x}_i\TT\vec{x}_j$
	\item degree $p$ polynomial: $k[\vec{x}_i,\vec{x}_j]=\left(\vec{x}_i\TT\vec{x}_j+1\right)^p$
	\item RBF or Gaussian:
	\begin{align*}
		k[\vec{x}_i,\vec{x}_j]=\e{-\frac{(\vec{x}_i-\vec{x}_j)\T(\vec{x}_i-\vec{x}_j)}{2\la^2}}
	\end{align*}
\end{enumerate}
Gaussian process regression using an RBF (Gaussian) kernel. The parameter $\la$ controls how smooth the function is.

\section{Sparse Linear Regression}

Sometimes not every dimension of the data $\vec{x}$ is informative. A sparse solution forces some of the coefficients in $\vec{\ph}$ to be zero. This can be done by applying product of t-distributions to parameter vectors
\begin{align*}
	\P{\vec{\ph}}&=\prod_{d=1}^D\Stud_{\ph_d}[0,1,\nu] \\
	&=\prod_{d=1}^D\frac{\Ga\left(\frac{\nu+1}{2}\right)}{\sqrt{\nu\pi}\Ga\left(\frac{\nu}{2}\right)}\left(1+\frac{\ph_d^2}{\nu}\right)^{-(\nu+1)/2}
\end{align*}
Using Bayes' rule, we can have the posterior distribution as
\begin{align*}
	\P{\vec{\ph}|\mat{X},\vec{w},\si^2}=\frac{\P{\vec{w}|\mat{X},\vec{\ph},\si^2}\P{\vec{\ph}}}{\P{\vec{w}|\mat{X},\si^2}}
\end{align*}
However, this time the prior is not conjugate to the normal likelihood so cannot compute posterior in closed form. To make progress, we can write t-distribution as marginal of joint distribution
\begin{align*}
	\P{\vec{\ph}}&=\prod_{d=1}^D\int\Norm_{\ph_d}\left[0,\frac{1}{h_d}\right]\Gam_{h_d}\left[\frac{\nu}{2},\frac{\nu}{2}\right]dh_d \\
	&=\int\Norm_{\vec{\ph}}\left[0,\mat{H}\inv\right]\prod_{d=1}^D\Gam_{h_d}\left[\frac{\nu}{2},\frac{\nu}{2}\right]d\mat{H}
\end{align*}
where $\mat{H}$ is a diagonal matrix with hidden variables $h_d$ on diagonal. Then we have
\begin{align*}
	\P{\vec{w}|\mat{X},\si^2}&\prop\int\P{\vec{w},\vec{\ph}|\mat{X},\si^2}d\vec{\ph} \\
	&=\int\P{\vec{w}|\mat{X},\vec{\ph},\si^2}\P{\vec{\ph}}d\vec{\ph} \\
	&=\int\Norm_{\vec{w}}\left[\mat{X}\TT\vec{\ph},\si^2\mat{I}\right]\int\Norm_{\vec{\ph}}\left[0,\mat{H}\inv\right]\prod_{d=1}^D\Gam_{h_d}\left[\frac{\nu}{2},\frac{\nu}{2}\right]d\mat{H}d\vec{\ph} \\
	&=\iint\Norm_{\vec{w}}\left[\mat{X}\TT\vec{\ph},\si^2\mat{I}\right]\Norm_{\vec{\ph}}\left[0,\mat{H}\inv\right]\prod_{d=1}^D\Gam_{h_d}\left[\frac{\nu}{2},\frac{\nu}{2}\right]d\mat{H}d\vec{\ph} \\
	&=\int\Norm_{\vec{w}}\left[\vec{0},\mat{X}\TT\mat{H}\inv\mat{X}+\si^2\mat{I}\right]\prod_{d=1}^D\Gam_{h_d}\left[\frac{\nu}{2},\frac{\nu}{2}\right]d\mat{H}
\end{align*}
We still cannot compute it, but we can approximate that
\begin{align*}
	\P{\vec{w}|\mat{X},\si^2}\approx\max_{\mat{H}}\left[\Norm_{\vec{w}}\left[\vec{0},\mat{X}\TT\mat{H}\inv\mat{X}+\si^2\mat{I}\right]\prod_{d=1}^D\Gam_{h_d}\left[\frac{\nu}{2},\frac{\nu}{2}\right]\right]
\end{align*}
To fit the model, update variance $\si^2$ and hidden variables $h_d$ according to the updates:
\begin{align*}
	h_d&\gets\frac{1-h_d\mat{\Si}_{dd}+\nu}{\mu_d^2+\nu} \\
	\si^2&\gets\frac{1}{D-\sum_d(1-h_d\mat{\Si}_{dd})}\left(\vec{w}-\mat{X}\vec{\mu}\right)\T\left(\vec{w}-\mat{X}\vec{\mu}\right)
\end{align*}
where
\begin{align*}
	\vec{\mu}&=\frac{1}{\si^2}\mat{A}\inv\mat{X}\vec{w} \\
	\mat{\Si}&=\mat{A}\inv \\
	\mat{A}&=\frac{1}{\si^2}\mat{X}\mat{X}\TT+\mat{H}
\end{align*}
After fitting, some of hidden variables become very big, implies prior tightly fitted around zero, can be eliminated from model. However it does not work for non-linear case as we need one hidden variable per dimension, which becomes intractable with high dimensional transformation.

\section{Dual Linear Regression}

We can represent $\vec{\ph}$ as a weighted sum of the data points that $\vec{\ph}=\mat{X}\vec{\ps}$, since gradient $\vec{\ph}$ is just a vector in the data space. Now we can instead solve for $\vec{\ps}$, one parameter per training example. Hence, we can write the likelihood term as
\begin{align*}
	\P{\vec{w}|\mat{X},\vec{\th}}&=\Norm_{\vec{w}}\left[\mat{X}\T\vec{\ph},\si^2\mat{I}\right] \\
	&=\Norm_{\vec{w}}\left[\mat{X}\T\mat{X}\vec{\ps},\si^2\mat{I}\right]
\end{align*}
Learning with maximum likelihood we have
\begin{align*}
	\hat{\vec{\ps}}&=\left(\mat{X}\TT\mat{X}\right)\inv\vec{w} \\
	\hat{\si}^2&=\frac{\left(\vec{w}-\mat{X}\TT\mat{X}\vec{\ps}\right)\T\left(\vec{w}-\mat{X}\TT\mat{X}\vec{\ps}\right)}{\mat{I}}
\end{align*}
It has the same result as before
\begin{align*}
	\hat{\vec{\ph}}=\mat{X}\hat{\vec{\ps}}&=\mat{X}\left(\mat{X}\TT\mat{X}\right)\inv\vec{w} \\
	&=\left(\mat{X}\mat{X}\TT\right)\inv\mat{X}\mat{X}\TT\mat{X}\left(\mat{X}\TT\mat{X}\right)\inv\vec{w} \\
	&=\left(\mat{X}\mat{X}\TT\right)\inv\mat{X}\vec{w}
\end{align*}

As before, in Bayesian inference, we give a prior to the parameter $\vec{\ps}$ that $\P{\vec{\ps}}=\Norm_{\vec{\ps}}[\vec{0},\si_p^2\mat{I}]$. Hence the posterior distribution is
\begin{align*}
	\P{\vec{\ps}|\mat{X},\vec{w},\si^2}&=\frac{\P{\vec{w}|\mat{X},\vec{\ps},\si^2}\P{\vec{\ps}}}{\P{\vec{w}|\mat{X},\si^2}} \\
	&=\Norm_{\vec{\ps}}\left[\frac{1}{\si^2}\mat{A}\inv\mat{X}\TT\mat{X}\vec{w},\mat{A}\inv\right]
\end{align*}
where
\begin{align*}
	\mat{A}=\frac{1}{\si^2}\mat{X}\TT\mat{X}\mat{X}\TT\mat{X}+\frac{1}{\si_p^2}\mat{I}
\end{align*}
And the predictive distribution is
\begin{align*}
	\P{w^*|\vec{x}^*,\mat{X},\vec{w}}&=\int\P{w^*|\vec{x}^*,\vec{\ps}}\P{\vec{\ps}|\mat{X},\vec{w}}d\vec{\ps} \\
	&=\Norm_{w^*}\left[\frac{1}{\si^2}\vec{x}^*\T\mat{X}\mat{A}\inv\mat{X}\TT\mat{X}\vec{w}^*,\vec{x}^*\T\mat{X}\mat{A}\inv\mat{X}\TT\vec{x}^*+\si^2\right]
\end{align*}
where
\begin{align*}
\mat{A}=\frac{1}{\si^2}\mat{X}\TT\mat{X}\mat{X}\TT\mat{X}+\frac{1}{\si_p^2}\mat{I}
\end{align*}
Notice that both the maximum likelihood and Bayesian inference depend on dot products $\mat{X}\TT\mat{X}$, which can be kernelized.

\section{Relevance Vector Regression}

We can combine the ideas of dual regression (one parameter per training example) and sparsity regression (most of the parameters are zero) to obtain a model that only depends sparsely on training data. In this case, we have our likelihood and prior term as
\begin{align*}
	\P{\vec{w}|\mat{X},\vec{\th}}&=\Norm_{\vec{w}}\left[\mat{X}\TT\mat{X}\vec{\ps},\si^2\mat{I}\right] \\
	\P{\vec{\ps}}&=\prod_{i=1}^N\Stud_{\ps_i}[0,1,\nu]
\end{align*}
Similarly, we use the approximations that
\begin{align*}
	\P{\vec{w}|\mat{X},\si^2}\approx\max_{\mat{H}}\left[\Norm_{\vec{w}}\left[\vec{0},\mat{X}\TT\mat{X}\mat{H}\inv\mat{X}\TT\mat{X}+\si^2\mat{I}\right]\prod_{d=1}^D\Gam_{h_d}\left[\frac{\nu}{2},\frac{\nu}{2}\right]\right]
\end{align*}
and the updates over parameters
\begin{align*}
	h_i&\gets\frac{1-h_i\mat{\Si}_{ii}+\nu}{\mu_i^2+\nu} \\
	\si^2&\gets\frac{1}{\mat{I}-\sum_i(1-h_i\mat{\Si}_{ii})}\left(\vec{w}-\mat{X}\TT\mat{X}\vec{\mu}\right)\T\left(\vec{w}-\mat{X}\TT\mat{X}\vec{\mu}\right)
\end{align*}
where
\begin{align*}
	\vec{\mu}&=\frac{1}{\si^2}\mat{A}\inv\mat{X}\TT\mat{X}\vec{w} \\
	\mat{\Si}&=\mat{A}\inv \\
	\mat{A}&=\frac{1}{\si^2}\mat{X}\TT\mat{X}\mat{X}\TT\mat{X}+\mat{H}
\end{align*}
Notice that this only depends on dot-products and so can be kernelized.