%!TEX root = ../notebook.tex
% Chapter 3

\chapter{Machine Learning Basics}
\label{chapter3}

\section{Regularization}

\subsection{Under-fitting \& Over-fitting}

\begin{description}[leftmargin=0cm]
\item[Under-fitting] If $N>D$ (e.g. 30 data points, 2 dimensions) we have more equations than unknowns: over-determined system. Input-output relations can only hold approximately.
\item[Over-fitting] If $N<D$ (e.g. 30points, 15265 dimensions) we have more unknowns than equations: under-determined system. Input-output equations hold exactly, but we are simply memorizing data.
\end{description}

\subsection{Bias \& Variance}

\begin{description}[leftmargin=0cm]
\item[High Bias \& Low Variance] A rigid model's (low complexity) performance is more predictable in the test set but the model may not be good even on the training set.
\item[Low Bias \& High Variance] A flexible model (high complexity) approximates the target function well in the training set but can ``overtrain'' and have poor performance on the test set.
\end{description}

\subsection{Vector Norm}

\begin{description}[leftmargin=0cm]
\item[L1, (``Manhattan'') norm] $\norm{\vec{w}}_1=\sum_{d=1}^D|w_d|$
\item[L2, (``Euclidean'') norm] $\norm{\vec{w}}_2=\sqrt{\sum_{d=1}^Dw_d^2}=\sqrt{\matmul{\vec{w},\vec{w}}}=\sqrt{\vec{w}\T\vec{w}}$
\item[Lp norm, p$>$1] $\norm{\vec{w}}_p=\left(\sum_{d=1}^Dw_d^p\right)^{1/p}$
\end{description}

\subsection{Penalize Complexity}

In linear regression, the residual vector is $\vec{\ep}=\vec{y}-\vec{\Ps}\vec{w}$. The loss function is $L(\vec{w})=\vec{\ep}\T\vec{\ep}$. We add a complexity term $R(\vec{w})=\norm{\vec{w}}_2=\vec{w}\T\vec{w}$ to the loss function. Hence, the original loss function becomes $L(\vec{w})=\vec{\ep}\T\vec{\ep}+\la\vec{w}\T\vec{w}$.

Without regularization, the loss function is $L(\vec{w})=\vec{\ep}\T\vec{\ep}$. Let $\grad L(\vec{w}^*)=0$, we have $\vec{w}^*=(\mat{X}\T\mat{X})\inv\mat{X}\T\vec{y}$.

With L2-regularization, the loss function is $L(\vec{w})=\vec{\ep}\T\vec{\ep}+\la\vec{w}\T\vec{w}$. Let $\grad L(\vec{w}^*)=0$, we have $\vec{w}^*=(\mat{X}\T\mat{X}+\la\mat{I})\inv\mat{X}\T\vec{y}$. The additional $\la\mat{I}$ makes the data matrix more robust to calculate inversion.

\section{Cross-Validation}

We can select hyperparameters with (cross-)validation. Cross-validation excludes part of the training data from parameter estimation, and use them only to predict the test error.

K-fold cross validation: split data set into K folds and each time train on (K-1) folds and valid on the remaining fold until all folds have been used as validation fold. The cross-validation error is the average of K validation errors. We pick hyperparameters that minimize cross-validation error.

\section{Bayesian Learning}

\subsection{Bayes' Rule Terminology}

Bayes' Rule:
	\begin{align*}
	\P{y|x}=\frac{\P{x|y}\P{y}}{\int\P{x|y}\P{y}dy}
	\end{align*}
\begin{description}[leftmargin=0cm]
	\item[Prior] $\P{y}$ what we know about $y$ before seeing $x$. In parameters learning we choose prior that is conjugate to likelihood.
	\item[Likelihood] $\P{x|y}$ propensity for observing a certain value of $x$ given a certain value of $y$.
	\item[Posterior] $\P{y|x}$ what we know about $y$ after seeing $x$. Posterior must have same form as conjugate prior distribution.
	\item[Evidence] $\int\P{x|y}\P{y}dy$ a constant to ensure that the LHS is a valid distribution. Posterior must be a distribution which implies that evidence equals to a constant $\ka$ from conjugate relation.
\end{description}

\subsection{Maximum Likelihood}

\begin{description}[leftmargin=0cm]
\item[Fitting] As the name suggests we find the parameters under which the data $\vec{x}_{1\dots I}$ are most likely. Here, we have assumed that data was independent.
	\begin{align*}
	\hat{\vec{\th}}&=\argmax_{\vec{\th}}\P{\vec{x}_{1\dots I}|\vec{\th}} \\
				   &=\argmax_{\vec{\th}}\prod_{i=1}^I\P{\vec{x}_i|\vec{\th}}
	\end{align*}
\item[Predictive Density] Evaluate new data point $\vec{x}^*$ under probability distribution $\P{\vec{x}^*|\hat{\vec{\th}}}$ with best parameters.
\end{description}

\subsection{Maximum a Posterior (MAP)}

\begin{description}[leftmargin=0cm]
\item[Fitting] As the name suggests we find the parameters which maximize the posterior probability $\P{\vec{\th}|\vec{x}_{1\dots I}}$. Again we have assumed that data was independent.
	\begin{align*}
	\hat{\vec{\th}}&=\argmax_{\vec{\th}}\P{\vec{\th}|\vec{x}_{1\dots I}} \\
				   &=\argmax_{\vec{\th}}\frac{\P{\vec{x}_{1\dots I}|\vec{\th}}\P{\vec{\th}}}{\P{\vec{x}_{1\dots I}}} \\
				   &=\argmax_{\vec{\th}}\frac{\prod_{i=1}^I\P{\vec{x}_i|\vec{\th}}\ \P{\vec{\th}}}{\P{\vec{x}_{1\dots I}}}
	\end{align*}
Since the denominator does not depend on the parameters we can instead maximize
	\begin{align*}
	\hat{\vec{\th}}=\argmax_{\vec{\th}}\prod_{i=1}^I\P{\vec{x}_i|\vec{\th}}\ \P{\vec{\th}}
	\end{align*}
\item[Predictive Density] Evaluate new data point $\vec{x}^*$ under probability distribution with MAP parameters $\P{\vec{x}^*|\hat{\vec{\th}}}$
\end{description}

\subsection{Bayesian Approach}

\begin{description}[leftmargin=0cm]
\item[Fitting] Compute the posterior distribution over possible parameter values using Bayes' rule. Principle: There are many values that could have explained the data. Instead of picking one set of parameters, try to capture all of the possibilities.
	\begin{align*}
	\P{\vec{\th}|\vec{x}_{1\dots I}}=\frac{\prod_{i=1}^I\P{\vec{x}_i|\vec{\th}}\ \P{\vec{\th}}}{\P{\vec{x}_{1\dots I}}}
	\end{align*}
\item[Predictive Density] (a) Each possible parameter value makes a prediction. (b) Some parameters more probable than others.
	\begin{align*}
	\P{\vec{x}^*|\vec{x}_{1\dots I}}=\int\P{\vec{x}^*|\vec{\th}}\P{\vec{\th}|\vec{x}_{1\dots I}}d\vec{\th}
	\end{align*}
Make a prediction that is an infinite weighted sum (integral) of the predictions for each parameter value ($\P{\vec{x}^*|\vec{\th}}$), where weights are the probabilities ($\P{\vec{\th}|\vec{x}_{1\dots I}}$).
\end{description}

\subsection{Example: Normal Distribution}

\subsection{Example: Categorical Distribution}

\section{Machine Learning Models}